\subsection{DataAccessService}
Der DataAccessService ist für die Kommunikation mit dem Model zuständig.
Nur über diesen können Daten abgefragt werden und gespeichert werden.
Dabei werden die Daten im Gegensatz zur View als richtige Pruefung und als richtiger Block gespeichert.
Das heißt, dass die Pruefungen kein Scoring besitzen und auch direkt verändert werden können.

