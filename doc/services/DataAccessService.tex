\subsection{DataAccessService}\label{subsec:dataaccessservice}
Der DataAccessService ist für die Kommunikation mit dem Model zuständig.
Nur über diesen können Daten abgefragt und gespeichert werden.
Dabei werden die Daten im Gegensatz zur View als richtige Prüfung und als richtiger Block gespeichert.
Das heißt, dass die Prüfungen kein Scoring besitzen und auch direkt verändert werden können.

