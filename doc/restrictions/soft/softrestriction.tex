\subsection{Weiche Restriktion}\label{subsec:weiche_restriktion}

\subsubsection{Was ist eine weiche Restriktion?}
Eine weiche Restriktion ist, ein Kriterium beim Einplanen von Klausuren in den Klausurplan.
Bei dieser Art von Restriktion wird im Gegenteil zu der harten Restriktion die Klausur eingeplant,
wenn dieses Kriterium fehlschlägt.
Aber dieses Kriterium verändert das Scoring von der Klausur und von allen anderen Klausuren, die von
der neu eingeplanten Klausur beeinflusst sind.

\subsubsection{Scoring}
Das Scoring bei einer Prüfung gibt an, wie gut eine Prüfung an dem Zeitslot liegt.
Dabei gibt es viele WeichesKriterien, die diesen Wert beeinflussen.
Hierbei kann ein WeichesKriterium auch das Scoring mehr als einmal verschlechtern, in dem das Kriterium
besonders stark verletzt ist.
\subsubsection{Mehrere Klausuren an einem Tag}
Dieses Kriterium ist dafür da zu überprüfen, ob ein Teilnehmerkreis mehrere Prüfungen
an einem Tag schreibt.
Falls dieses Kriterium verletzt ist, wird das Scoring von allen beteiligten Prüfungen stark negativ
beeinflusst.
Die genaue Auswirkung auf das Scoring nimmt mit der Anzahl der betroffenen Studenten zu.

\subsection{Anzahl Prüfung pro Woche}
Der Kunde bzw. die Kundin wünscht sich,
dass eine bestimmte Anzahl von Prüfungen für ein gewissen Teilnehmerkreis
in der Woche nicht überschritten werden.
Hierbei wird für eine übergebenen Prüfung inspiziert,
ob diese Prüfung Teinehmerkreise beinhaltet, dessen Anzahl der zu schreibenden Klausuren in der selben Woche,
wie der zu überprüfenden Prüfung das Limit von 4 übersteigen.
Zuerst werden die geplanten Prüfungen nach Wochenabstand zum Startdatum der Prüfungsperiode sortiert.
Anschließend wird für jeden Teilnehmerkreis der übergebenen Prüfung kontrolliert,
ob die Restriktion verletzt wird.
Dabei wird jeder Teilnehmerkreis der zu überprüfenden Prüfung mit
den Prüfungen verglichen, die in der selben Woche wie der der zu überprüfenden Prüfung stattfinden.
Prüfungen die sich im selben Block wie in der zu überprüfenden Prüfung befinden,
werden hierbei ignoriert und nicht beachtet.
Für jede Verletzung der Restriktion durch einen Teilnehmerkreises der übergebenen Prüfung wird das Scoring der Prüfung erhöht.




\subsubsection{Die 4. Woche der Periode für Masterstudierende}
Eines der Restriktionen ist es, dass in der vierten Woche der Prüfungsperiode,
ausgehend vom Startdatum, nur Prüfungen mit Master Teilnehmerkreisen geschrieben werden.
Prüfungen, die Master und Bachelor Teilnehmerkreise beinhalten, dürfen auch in der vierten Woche
geschrieben werden.
Die Restriktion wird dann verletzt, wenn eine Prüfung nur Bachelor Teilnehmerkreise (oder auch PTL) beinhaltet und in der 4. Woche stattfindet.


\subsubsection{Keine Klausur am Sonntag}
Laut dem Kunden bzw.\ der Kundin, soll keine Klausur am Sonntag stattfinden.
Die Restriktion wird für eine Klausur dann verletzt, wenn sie am Sonntag stattfindet.
\subsubsection{Anzahl gleichzeitig zu hoch}
Die Klasse \enquote{AtSameTimeRestriction} dient als eine Zusammenfassung der grundlegenden
gemeinsamen operationen zum Ermitteln von Verletzungen, welche die Gleichzeitigkeit von Pruefungen
zur Bedingung haben.

Konkrete Implementierungen hierzu sind
\hyperref[subsubsec:nicht-zu-viele-pruefungen-gleichzeitig]{\enquote{Nicht zu viele Prüfungen gleichzeitig}}
und
\hyperref[subsubsec:nicht-zu-viele-teilnehmer-gleichzeitig]{\enquote{Nicht zu viele Teilnehmer gleichzeitig}}

\subsubsection{Nicht zu viele Prüfungen gleichzeitig}\label{subsubsec:nicht-zu-viele-pruefungen-gleichzeitig}
Die Klasse \enquote{AnzahlPruefungenGleichzeitigRestriction} beschreibt eine konkrete Implementierung von
\enquote{\hyperref[subsubsec:anzahl-gleichzeitig-zu-hoch]{Anzahl gleichzeitig zu hoch}}.
Sie stellt eine Restriktion dar, welche eine Obergrenze für die Anzahl als gleichzeitig einzuplanende
Planungseinheiten darstellt.
Der Ausdruck der Gleichzeitigkeit wird in diesem Zusammenhang um einen Pufferzeitraum
um Planungseinheiten herum erweitert um ein Vor- und Nachbereiten von Planungseinheiten sicherzustellen.

\subsubsection{Nicht zu viele Teilnehmer gleichzeitig}\label{subsubsec:nicht-zu-viele-teilnehmer-gleichzeitig}
Die Klasse \enquote{AnzahlPruefungenGleichzeitigRestriction} beschreibt eine konkrete Implementierung von
\hyperref[subsubsec:anzahl-gleichzeitig-zu-hoch]{\enquote{\enquoute{Anzahl gleichzeitig zu hoch}}}.
Sie stellt eine Restriktion dar, welche eine Obergrenze für die Anzahl an Studenten,
die zur selben Zeit an einer Prüfung teilnehmen, darstellt.
Der Ausdruck der Gleichzeitigkeit wird in diesem Zusammenhang um einen Pufferzeitraum
um Planungseinheiten herum erweitert um ein Vor- und Nachbereiten von Planungseinheiten sicherzustellen.

\subsubsection{Prüfungen mit vielen am Anfang}\label{subsubsec:pruefungenMitVielenAmAnfang}
Die Klasse „PruefungenMitVielenAmAnfangRestriction“ bildet die Weiche Restriktion „PRUEFUNGEN\_MIT\_VIELEN\_AN\_ANFANG“ ab.
Diese Restriktion ist dafür da, das Prüfungen mit einer Anzahl von Teilnehmer über ein bestimmten Wert( aktuell sind es 100) in den ersten sieben Tagen nach dem Ankertag liegt.
Dies beeinflusst das Scoring nur leicht.

\subsubsection{Uniforme Zeitslots}\label{subsubsec:uniformeZeitslots}
Die Klasse „UniformeZeitslotsRestriction“ bildet die Weiche Restriktion UNIFORME\_ZEITSLOTS ab.
Diese Restriktion ist dafür da, dass wenn eine Prüfung zu irgendeinen Zeitpunkt gleichzeitig mit einer anderen Prüfung liegt, diese Prüfungen dann die gleiche Länge besitzen müssen.
Dieses beeinflusst das Scoring mittelmäßig.

\subsubsection{Freier Tag zwischen Prüfungen}\label{subsubsec:freierTagZwischenPruefungen}
