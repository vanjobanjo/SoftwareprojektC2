\subsection{Integrationstests}\label{subsec:integrationstests}
Um das korrekte Zusammenwirken aller Komponenten und ein Einhalten der,
durch das Controller-Interface definierten, Zusagen an Nutzer des Moduls sicherzustellen,
existiert eine Reihe von Integrationstests.

Bei der Umsetzung dieser Integrationstests wird ein besonderer Fokus auf eine intuitive
Nachvollziehbarkeit der Funktionen gesetzt indem das Framework \enquote{Cucumber} eingesetzt wird.
Cucumber bietet die Möglichkeit Tests im BDD (Behaviour Driven Development) Stil zu erstellen und
damit eine so genannte \enquote{specification by example} zu bieten.
Dies äußert sich konkret an der Nutzung der Sprache \enquote{Gherkin}, welche sich dadurch auszeichnet
programmatische Logik hinter ausdrucksstarken, natürlichen Sätze zu abstrahieren.
Das bedeutet, dass ein Testfall zunächst in einfachen Sätzen, welche nicht nur für Programmierer,
sondern auch für Domänenexperten verständlich sind erstellt wird und im Hintergrund zu den
jeweiligen Sätzen eine Implementierung in Form von Code folgt.

Dabei wird mit dem BDD typischen muster \enquote{Given, When, Then} (zu deutsch: \enquote{Angenommen, Wenn, Dann})
gearbeitet.
Der Punkt \enquote{Angenommen} bildet dabei den ersten Schritt, er beschreibt den Zustand in dem sich
das zu testende System befindet, bevor das gewünschte Verhalten getestet wird.
Darauf folgt der \enquote{Wenn} Teil, darin wird die Aktion beschrieben, welche Gegenstand des Tests ist.
Abschließend folgt der \enquote{Dann} Teil, dieser beschreibt das Resultat, welches nach der zuvor beschriebenen
Aktion im beschriebenen Zustand erwartet wird.

Der Zusammenschluss dieser Schritte beschreibt also ein Testszenario, ein Beispiel dafür, wie sich
das System verhalten Soll.
Mehrere dieser Szenarien werden wiederum zu Funktionalitäten zusammengeschlossen und in einer
gemeinsamen Datei abgelegt, welche eine \enquote{.feature} Endung aufweist.

Der dahinter liegende Code (so genannte Schritt-Definitionen) wird in dedizierten Java Klassen,
nach Funktionalität gruppiert, abgelegt und eine Vebindung zwischen den Dateien über
framework speziefische Annotationen in den Schritt-Definitionen realisiert.
